\documentclass[a4paper, 12pt]{article}
\usepackage[
    type={CC},
    modifier={by},
    version={4.0},
]{doclicense}                
\setlength{\parindent}{6mm}
\usepackage{amsmath,amsthm,amssymb,amsfonts, mathtools}
\usepackage{graphicx}
\usepackage[margin=1.5in]{geometry}    % For margin alignment
\usepackage[english]{babel}
\usepackage[utf8]{inputenc}
\usepackage{enumerate}
\usepackage[shortlabels]{enumitem}
\usepackage{pdfpages}
\usepackage{arevmath}     % For math symbols
\usepackage{exercises}
\usepackage[ruled, lined, linesnumbered, commentsnumbered, longend]{algorithm2e}
\DeclarePairedDelimiter{\ceil}{\lceil}{\rceil}
\SetKwInput{KwInput}{Input}                % Set the Input
\SetKwInput{KwOutput}{Output}              % set the Output
\usepackage[
backend=biber,
style=alphabetic,
sorting=ynt
]{biblatex}
\addbibresource{source.bib}

\usepackage{fancyhdr}
\pagestyle{fancy}
\fancyhf{}
\fancyhead[LE,RO]{Alberto Defendi}
\fancyhead[RE,LO]{IoT: week 2 solutions}
\fancyfoot[LE,RO]{\thepage}

\title{Problem solutions 2}
\author{Alberto Defendi - alberto.defendi@helsinki.fi}
\date{\today}
\begin{document}
\maketitle

\section{Solutions} % (fold)
\label{sec:solutions}
\begin{exercise}
We consider the analysis of a self-driving car of autonomy level-1, where the
driver is assisted with an assistance system for steering or
acceleration/deceleration \cite{car_levels_international_2014}.
\begin{enumerate}
    \item 
		\begin{description}
		    \item[Event based] Level-1 requires intervention if the 
				autonomous system fails to respond properly, so
				it must be possible to take control of the driving process by
				acting on the speed level if the system fails to adjust the car
				to a suggested speed. This is a sensible aspect of the system,
				since in a possible failure, the car could reach speeds that may 
				make it difficult to control, or may break the law.\\
				Reading data from the sensors and evaluate car's condition can
				be done using event-based, since it would increase the accuracy
				of the measurements' timing.
			\item[Thread-based] Check of data from GPS, street maps and police
				database should run
				asynchronously without the intervention of the pilot, because
				the system may consult internet or access the main memory to
				check the data, which can be a slow operation.
		\end{description}
	\item Preemptive scheduling, because we can have 
		processes of higher priority, 
		like the action of the pilot on the level, that would
		switch from the assisted driving thread, to the manual. This would
		require a more complex scheduler, but this is balanced by the fact that 
		the car is equipped with a powerful computer that is also used for
		infotainment.
	\item In this car we can have a on-board computer that acts a
		a gateway, manages infotainment and does computation from sensor's
		data, and multiple low-end devices that collect data from sensors and do
		small computations to check if the measurements are in the normal range.
		The computational power of this device may be similar to a high-range
		ARM-based tablet that is power-efficient in fall 2021.
		The gateway OS should be: 
		\begin{itemize}
		    \item Multi-threading.
			\item Offer a GUI compatibility.
			\item Support drivers for interacting with internal peripherals 
				(e.g touchscreen) interacting with external peripherals (low-end
				devices) and radio/WiFi connectivity.
			\item Security features to avoid the auto to get hacked.
			\item A suitable OS could be Android.
			\item Dynamic memory, since there can be many application. The
				fragmentation is not problematic, since the main memory is a
				fast flash memory. 
		\end{itemize}
		The self-driving module OS should support: 
		\begin{itemize}
			\item Real time constraint (see next point).
			\item Events. 
		\end{itemize}
	\item 
		\begin{description}
		    \item[Hard] If the driver cannot take control of the car, it results
				in a system failure and the car should stop gracefully, without
				becoming dangerous for other street users. Also if the sensors
				that collect data from the car devices fail, 
			\item[Firm] If the autonomous system fails to collect speed data to
				assist the speed change, the driver should be informed and the
				system should become fully manual. 
			\item[Soft] A possible soft constraint in the car, but not related
				to the driving experience, can be the live audio streaming of
				the infotainment, which down-samples to a lower bit-rate if there
				is low signal or if the system is at computational capacity
				level. 
		\end{description}
	\item 
		\begin{description}
		    \item[Always on] The car has the advantage of having a discrete
				amount of energy autonomy, which can be used to power computers.
			Elements that are involved in the car driving must always be kept
			on to avoid incidents. Also the car condition sensors should
			continuously monitor possible issues. 
		\item[Core sleep] The devices core can be splitted in high-power and
			low-power core, the firsts used where the demand of computations is
			higher, the seconds where is lower.
			\item[Screen dim] The infotainment screen can be dimmed to reduce
				energy consumption.
		\end{description}
\end{enumerate}
\end{exercise}

\begin{exercise}
	\subsection{Consumer: Smart body scale} % (fold)
	\label{sub:consumer_smart_body_scale}
	\begin{enumerate}[(a)]
	    \item Weights people body mass and provides data analysis
			trough a mobile application. This helps the user to keep track of the
			weekly/monthly average and of any weight change. 
		\item
			\begin{description}
			    \item[Peripherals] The device should support peripherals such
					as a weight sensor, a LCD, a set of buttons and a WiFi
					antenna.
				\item[Connection] The data is visualized trough a display, and is possible to select
			the current user from a set of buttons on the device. 
			The device can upload measurements on a centralized trough the WiFi
			connection, but if this results to be too expensive to produce, the
			scale can connect to a phone trough Bluetooth and use the phone as a
			gateway.
				\item[Deep sleep] The device operation is fully event-driven,
					since all the operations are executed during the usage of
					the balance, and when not used, the device goes to deep
					sleep. An example flow could be: the user selects its
					profile, steps on the balance and waits until the result,
					the data is sent to the server and the device goes to sleep.  
				\item[Non-preemptive] Since the operations can happen
					sequentially, there is no need to have higher priority
					processes, as long as the scale firsts measures the person,
					and then sends the data (which may take longer).
				\item[Event-based] The scale could wake up when a change on the
					weight sensor is detected. After that, the data is
					visualized and sent.
				\item[Static allocation] The allocation can be static since the
					data of the program is the measurement of the person, that
					is later sent and deleted.
			\end{description}
		\item Tiny OS would be the best choice in this situation
			where the scale integrates a low-end device with little power (such
			as an array of 3/4 AA batteries), little memory and a limited number
			of actions.
	\end{enumerate}
	
	% subsection consumer_smart_body_scale (end)

\subsection{Enterprise: Room management} % (fold)
\label{sub:enterprise_room_management}
\begin{enumerate}[(a)]
    \item We develop an IoT system to offer a smart monitor to
		view and track the availability of a room, showing the event calendar
		and the possibility to reserve a room. A possible application could
		be adding one of these device outside a classroom at the University of
		Helsinki, where the daily lecture calendar is replaced manually
		everyday. The final system, will be integrated with the already existent
		info-system and will make possible to view the lessons calendar in that room
		and add the possibility to reserve a room
		if needed trough an online portal hosted by a gateway. 
	\item
		\begin{description}
		    \item[Peripherals] Should support drivers for epaper display and 
				an Ethernet connection.
			\item[Price] The device has to be low-end, since the computations are 
				minimal and simple. 
			\item[Low power] The control unit energy requirements can be reduced
				at minimum, since the operations are infrequent over time (the
				device syncs only when the gateway (a server) updates the
				timetable). It would be efficient to have the device in
				deep-sleep when no events are happening, since if we are using
				a e-paper display, it can hold static text
				and images indefinitely without electricity. The on trigger can
				be done when any signal is received from the gateway from the
				LAN cable.
			\item[Memory] Small memory is needed, enough to host the system and
				to temporarily download the data before printing it on the
				display. It is important that the program deletes the data, to
				avoid to run out of memory.
			\item[Non-preemptive] Since the limited power, some tasks can take
				long, it is important to give time to each task to finish.
				This design is easier to implement and requires a simpler
				mechanism to context-switch.
			\item[Static allocation] The data size is limited to what can fit on
				a display, and the system does only the simple task of fetching
				data. 
			\item[Security] The device must avoid the connection with unwanted
				host, because it could be used to breach into the network.
		\end{description}
	\item Tiny OS would be perfect, since the device is low-end, receive the
		power trough the LAN cable and should do only simple operations.
\end{enumerate}



% subsection enterprise_room_management (end)
    
\end{exercise}

\begin{exercise}
\subsection{RIOT OS} % (fold)
\label{sub:RIOT_oS}
The OS of my choice is RIOT \cite{8320780}, specifically designed for IoT devices and
lightweight in size, without sacrificing features like multi-threading.

\begin{description}
    \item[Architecture] Unlike Tiny OS which is monolithic, RIOT implements a
		micro-kernel design, which supports multi-threading and networking
		communications. The kernel supports real-time preemptive tickless scheduling, where
		the scheduler is run only when a thread is unlocked (saving additional
		power). \\
		The kernel architecture is adapted from Fire kernel, which provides
		primary features such as scheduling, process communication and
		synchronization. 
	\item[Advantages:]
		\begin{description}
		    \item[6LoWPAN] The OS supports 6LoWPAN, which enables 
				low-power radio communication between multiple devices.
			\item[Low-memory usage] Thanks to its micro-kernel, the OS has a
				low impact on the memory usage, using the size of just 1.5KB RAM
				and 5KB of ROM.  
			\item[Developer friendly] Implements multi-threading, is effective
				in memory, in terms of APIs (partial POSIX compliance) and is robust against bugs.
			\item[POSIX] The partial POSIX compliance makes possible to have an
				uniform API that is independent of underlying hardware, this
				makes possible to use RIOT with boards using different types of
				instruction sets. The documentation says that RIOT supports 100+
				boards \cite{riot-os_2021}.
			\item[Micro-kernel] This kernel architecture can help reduce the
				kernel size in small OSs like RIOT. 
			\item[Linear time fast execution] As reported by the article, the
				task execution has an efficiency of $O(1)$.
			\item[Open source] and relatively small code-base, thanks to the
				micro-kernel architecture, making it easier to contribute for
				aspiring OS developers, compared to monolithic OS.  
		\end{description}
	\item[Limitations:]
		\begin{description}
		    \item[Language support] Due to its tiny size, the OS SDK only
				supports the languages it was programmed in, which are C and
				C++. Despite those language being popular nowadays, new appealing
				alternatives for low-level systems are growing, 
				such as GO and Rust, which offer low-level access and can be easier
				to use.
			\item[POSIX] At the moment of writing, the OS only partially
				supports POSIX standard API, which can be problematic for some
				application and may require maintenance problems in future.
			\item[Micro-kernel] The micro-kernel architecture can have security  
			\item[No concurrency]
			\item[No security features]
			\item[No database]
			\item[No documentation]
		\end{description}
	\item[Applications] RIOT OS is applicable anywhere where low-end devices are
		used (where the memory, battery and power of devices is a constraint).
		The OS can be used in real-time applications.
\end{description}



% subsection RIOT_oS (end)

\end{exercise}


% section solutions (end)

\medskip

\printbibliography

\doclicenseThis

\end{document}
