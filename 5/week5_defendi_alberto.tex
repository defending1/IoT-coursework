\documentclass[a4paper, 12pt]{article}
\usepackage[
    type={CC},
    modifier={by},
    version={4.0},
]{doclicense}                
\setlength{\parindent}{6mm}
\usepackage{amsmath,amsthm,amssymb,amsfonts, mathtools}
\usepackage{graphicx}
\usepackage[margin=1.5in]{geometry}    % For margin alignment
\usepackage[english]{babel}
\usepackage[utf8]{inputenc}
\usepackage{enumerate}
\usepackage[shortlabels]{enumitem}
\usepackage{pdfpages}
\usepackage[ruled, lined, linesnumbered, commentsnumbered, longend]{algorithm2e}
\usepackage{arevmath}     % For math symbols
\usepackage{exercises}
\usepackage[ruled, lined, linesnumbered, commentsnumbered, longend]{algorithm2e}
\DeclarePairedDelimiter{\ceil}{\lceil}{\rceil}
\SetKwInput{KwInput}{Input}                % Set the Input
\SetKwInput{KwOutput}{Output}              % set the Output
\usepackage[
backend=biber,
style=alphabetic,
sorting=ynt
]{biblatex}
\addbibresource{source.bib}

\usepackage{fancyhdr}
\pagestyle{fancy}
\fancyhf{}
\fancyhead[LE,RO]{Alberto Defendi}
\fancyhead[RE,LO]{IoT: week 5 solutions}
\fancyfoot[LE,RO]{\thepage}

\title{Problem solutions 5}
\author{Alberto Defendi - alberto.defendi@helsinki.fi}
\date{\today}
\begin{document}
\maketitle

\section{Solutions} % (fold)
\label{sec:solutions}

\begin{exercise}
\begin{enumerate}
    \item 
		\begin{verbatim}
		ON EVENT fire-detect(loc):
		SELECT AVG(temp), AVG(co2), event 
		FROM SENSORS AS TMP 
		WHERE dist(tmp.temp, event.tmp) > 2 AND  dist(tmp.co2, event.co2) > 3
		SAMPLE PERIOD 2s FOR 30s;
		\end{verbatim}
	\item
		\begin{verbatim}
		    SELECT AVG(CO2), AVG(temp) 
			FROM SENSORS AS tmp 
			WHERE LOCATION = loc 
			AND tmp.temp > threshold.temp AND tmp.co2 >
			threshold.co2
			FRESHNESS 10 seconds;
		\end{verbatim}
	\item To support event based queries, the system should support event-based
		programming. To support lifetime-based queries, the system should
		should support thread-based programming.
	\item It should be made sure that the new devices uses the same operating
		system (possibly the same version). First, the administrator should check if
		there are drivers for the control unit for this OS, and check if the
		sensors record correct values.
\end{enumerate}
\end{exercise}
\begin{exercise}
\begin{figure}
    \centering
    \def\svgwidth{\linewidth}
    \input{2.pdf_tex} 
    \caption{Smart car}
    \label{fig:inkscape_figure}
\end{figure}
\begin{enumerate}
    \item The tasks related to improve safety and driving quality should be performed
		on device to reduce latency. This can be done since the car PC has good
		energy resources and computational power. The data and the results from 
		offline training can be sent online to the cloud that improves the
		upcoming models and finds abnormal situation. This is advantageous
		because data collected from other users can be used to spot potential problems. 
	\item See fig. 1.
	\item 
		\begin{description}
		    \item[Locally] Sensor recordings and value checking. 
				As task sensors
				record values (such as measuring tires pressure) and signal the
			 gateway if the value respects a certain threshold. 
		 \item[Gateway] The gateway is the car's computer, which can perform some
			 analysis with the data coming from the sensors to trigger events.
			 Some tasks can be:
			 \begin{itemize}
			     \item Query data from sensors.
				 \item Use data from other programs (e.g cruise control,
					 collision detect, car status).
				\item Send data to server.
				\item Re-calibrate sensors.
			 \end{itemize}
		 \item[Cloud] The cloud does more complex calculation and data analysis.
			 Some tasks can be:
			 \begin{itemize}
			     \item Listen to car's update.
				\item Train data and provide analysis for other applications,
					such as traffic analysis, Geo-based alerts, fault detection. 
				\item Send the knowledge back to the car.
				\item Organize and store data.
			 \end{itemize}
		\end{description}
	\item Trough a P2P network, where each car gateway is a node of the network.
		This can only made possible if the manufacturers adopt the service and
		contribute to the network. The efficiency of the network depends from the
		number of cars around. 
\end{enumerate}
\end{exercise}
\begin{exercise}
\begin{enumerate}
    \item Warehouse or any centralized model, since a decentralized requires
		constant communication with the devices, and Sigfox has a limited daily
		exchange of 4 messages in down-link.
	\item It would be easier with LoRa class A, with Class C the receiver can
	receive data from the server only when the connection is open, which makes
	hard to have constant updates from the devices.
	\item 
		\begin{description}
		    \item[Value] it is important to determine how the detection of
				animal movement beneficial to the company. The system can be
				used to detect the preferred places of the animals, and to
				improve the spaces from the analysis.
			\item[Volume] it is important to establish an interval for the data
				sampling and to optimize the sensor's position to reduce the
				volume of data, which grows linearly with the formula
				$data_{minute} = number_{sensors} \cdot
				number_{recordings/min} \cdot data_{size}$
			\item[Variety] The data should the greatest amount of field,
				otherwise the measurements would be imprecise. 
		\end{description}
	\item 
		\begin{description}
		    \item[Real time stream] checking if the animals are currently in a
				certain zone. 
			\item[Batch] analysis on the zones preferred by the
				animals.  
		\end{description}
\end{enumerate}
\end{exercise}
\begin{exercise}
	\begin{enumerate}[(a)]
    \item 
		\begin{enumerate}[i.]
			\item Each node can have an hourly average entry in the table. When
				a sensor records the temperature, the sensor computes the average
				summing the value with the measurements recorded in the current
				hour, and saves it into the table.
				\begin{verbatim}
				    Average
					-------------
					+ average (int)
					+ timestamp (datetime)
				\end{verbatim}
				Where for each measurement $m$, $average_{h} = 
		      \frac{\sum_{i=1}^{n} m_{i}} {n} $. 
			\item The edges have the job of measuring temperature, and the tasks
				are  collecting data, inserting it into
				the database in a ordered data structure, run the
				measurement every specific
				interval of time, and compute the average. The client device runs
				the program to visualize and
				find the location with highest average, its task are querying the sensors every hour
				with a lifetime-based query, send the data in a larger storage,
				run the algorithm to find the max value, and display the data.
		    \item We can use an ordered  data structure such as a max-heap,
				which keeps the highest average as parent node. We can fetch 
				this element in $O(1)$ for every node.
 
		\end{enumerate}
	\item 
		\begin{enumerate}[(i)]
		\begin{figure}
			\centering
			\def\svgwidth{\linewidth}
			\input{6b.pdf_tex} 
			\caption{Temperature monitoring, three zones}
			\label{fig:inkscape_figure}
		\end{figure}
		    \item For each zone, the system is identical to 1, with the
				difference that each zone has a gateway that is liked with the
				edges in the area. Every area is independent. The gateway are
				interconnected in a secondary P2P network, hosting a distributed
				database that keeps the tables
				updated with the values from different zones. When users queries
				the nearest gateway, they can view values from all the zones. 
			\item The database should be distributed, but in my implementation
				is an extension of the area model, hence it just needed to
				create another network of coordinator formed by the gateways.
				A distributed DB is better because the gateway can act as a
				unique authority, and there is no need to insert a central warehouse as
				reference to send the data.
			\item 
				\begin{description}
				    \item[Volume] it should be defined the sampling time for
						recording the temperature, as there might be brusque
						changes in short intervals.
					\item[Variety] The variety has to be limited to the
						extremely necessary, namely the temperatures from the
						three zones of the city.
					\item[Velocity] The data should arrive in a short amount to
						provide the users weather forecasting.
					\item[Veracity] The sensors must record correct data and
						the platform must keep the value unvaried.  
				\end{description}
		\end{enumerate}
\end{enumerate}
\end{exercise}

% section solutions (end)

\medskip

\printbibliography

\doclicenseThis

\end{document}
