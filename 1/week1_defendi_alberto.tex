\documentclass[a4paper, 12pt]{article}
\usepackage[
    type={CC},
    modifier={by},
    version={4.0},
]{doclicense}                
\setlength{\parindent}{6mm}
\usepackage{amsmath,amsthm,amssymb,amsfonts, mathtools}
\usepackage{graphicx}
\usepackage[margin=1.5in]{geometry}    % For margin alignment
\usepackage[english]{babel}
\usepackage[utf8]{inputenc}
\usepackage{enumerate}
\usepackage[shortlabels]{enumitem}
\usepackage{pdfpages}
\usepackage{arevmath}     % For math symbols
\usepackage[ruled, lined, linesnumbered, commentsnumbered, longend]{algorithm2e}
\DeclarePairedDelimiter{\ceil}{\lceil}{\rceil}
\SetKwInput{KwInput}{Input}                % Set the Input
\SetKwInput{KwOutput}{Output}              % set the Output

\usepackage{fancyhdr}
\pagestyle{fancy}
\fancyhf{}
\fancyhead[LE,RO]{Alberto Defendi}
\fancyhead[RE,LO]{IoT: week 1 solutions}
\fancyfoot[LE,RO]{\thepage}

\usepackage[
backend=biber,
style=alphabetic,
sorting=ynt
]{biblatex}
\addbibresource{source.bib}


\title{Problem solutions 1}
\author{Alberto Defendi - alberto.defendi@helsinki.fi}
\date{\today}
\begin{document}
\maketitle

\section{} % (fold)
\label{sec:}
\subsection{} % (fold)
\label{sub:}

\begin{description}
	\item[Adressability] the car has a unique vehicle identification number (VIN)
	\item[Connectivity] the device has to be wired to exchange data with the
		external world, for now it's not an IoT. This could be extended by
		adding a radio module, or by using the connectivity from the user
		smartphone. 
	\item[Sensing] the car is equipped with many sensors and different knowledge
		can be extracted from those.
    \item[Programmability] having a computer on board can makes possible to use
		a programming language to automate task and deliver information to the
		passenger on the cockpit and to notify for the insurgence of problems. 
	\item[Future] The car could be turned into IoT by connecting it with a
		server and providing knowledge to a smartphone app, but this depends
		upon if the car was designed to be extended and if the manufacturer
		provides the extension modules to the customer, otherwise DIY solutions
		provide many communications. 
	\item[Privacy] an hacker could access the location of the car, causing
		serious repercussions on the car and person safety. Fake
		errors can be generated, or someone could re-calibrate the car and alter 
		the data values to generate fake alerts and fake evaluations. 
		It would be possible to develop an interface between the car and the
		smartphone to view the current location and unlock it from the
		smartphone, in this case the access should be protected with strong
		access control, to avoid someone to steal the car.  
\end{description}

% subsection  (end)

\subsection{} % (fold)
\label{sub:}
\begin{description}
    \item[Addressability] the fridge has an unique factory number, also the mac
		address/serial number of the computer could be used
	\item[Connectivity] the fridge should be placed in a part of the house with
		good WiFi connection or an Ethernet cable to work. A fast connection may
		be required if the camera resolution is high. 
	\item[Programmability] since it has a touchscreen, the computer should have
		a mobile operative system (e.g.. android) and support the most common
		high level programming languages.
	\item[Sensing] the camera can check if the fridge is empty (and if infrared,
		control the temperature of points in the fridge). The microphone/tablet
		are good interfaces to the user.
	\item[Future]
		The manufacturer
		could add temperature, humidity sensors to inform about the state of
		the fridge,
a gas sensor (for smell detection), an ice sensor (says if the
		ice is ready), an IR sensor (eggs detection). But new models shall be
		released if the manufacturer didn't predisposed extensible modules.
		\href{https://www.ijariit.com/manuscripts/v3i1/V3I1-1375.pdf}{Reference:}
	\item[Privacy] an hacker could take control trough the OS of the microphone/camera
		to spy John's private life, and
		the data of the products in the fridge could be sold to target John's
		preferences without his consent. The fridge tablet could be equipped with an app store,
		where could be possible to download video-conferencing tools, so that
		John could talk with someone from the fridge. Here there could be
		fraudulent apps that spied John camera/microphone.
\end{description}

% subsection  (end)

% section  (end)

\section{Smart Traffic} % (fold)
\label{sec:smart_traffic}
\subsection{Usage} % (fold)
\label{sub:usage}
The reality that I chose is related to my observation of the public transport
service in the metropolitan area of Helsinki. The methodology I used is trying
to reverse engineering the transport network to understand the mechanism, since
the documentation is not available. This research will focus
on the bus vehicles in the service zones (ABCD), while other vehicles are
behind our scope. 
We use \cite{6714496} and
\cite{8320780} as sample.
HSL offers a route planner that is accessible trough the website
\href{https://reittiopas.hsl.fi/etusivu}{Journey planner} and which uses data
from OpenStreetMap for the maps and data from their vehicles (bus, ferries,
trains) to feed a Journey planner (build upon OpenTripPlanner) to compute the
arrival time of the next vehicle, and offer a prediction of the best route to 
the passenger. 

% subsection usage (end)

\subsection{Benefits} % (fold)
\label{sub:benefits}
In pre-IoT times, the routes timetable was communicated trough the website or
with a schedule printed in the bus stop. However, this model is not flexible
enough in situations of traffic or when there are disruptions of services.\\
The advantages of this IoT network, enables the commuter to find faster routes
easily with the aid of the knowledge base extracted from the sensors' data. 
Having a journey planner makes easier to find the fastest path to reach a destination,
while suggesting alternative routes in case of disruption of services or when
using multiple routes is required to reach a destination. 
The interface to the service happens trough the use of a website or a mobile
app, where is also possible to purchase tickets for the public transport. 

% subsection benefits (end)

\subsection{Architecture} % (fold)
\label{sub:architecture}
The architectural components are:
\begin{itemize}
    \item bus receiver 
	\item stop transmitter
	\item on-board computer (gateway)
\end{itemize}
From my deductions, the implementation doesn't use the GPS for establishing the
position of the bus, which can
have connection problems in places where the signal is obstructed by
objects, such as tall buildings in the city center and also could require more
architectural complexity.\\
Instead, every route has different points (real bus stops), and a signal is
triggered when the bus passes at every point $S_{n}$, where $n$ is the number of
stops in a route. The bus keeps track of the time spent to reach point
$S_{i+1}$ from point $S_{i}$, and stores the time to adjust the prediction
algorithm, which we will not analyze, but supposing that it's Ford-Fulkerson or
Dijkstra's, it could be based on a weighted graph, where the weights is the time
needed to reach every stop.

% subsection architecture (end)


These are the three smart objects under analysis. We use:
\begin{description}
	\item[Bus state transmitter], which is located in the top part of the bus 
		and is equipped with a small receiver that listens to the nearby stop.
		\begin{description}
			\item[The function is] listening to nearby stops to detect when the
				bus has reached a stop, and send it to the gateway 
				(the driver computer), to update the current position of 
				the bus. 
			\item[The control unit] is similar to the one used in bus stops (a
				small low-power sensor controller) with the
				difference that this is powered by the bus engine and constantly
				exchanges data with the on-board PC.
			\item[Power source] Bus current.
		\end{description}
    \item[Bus stop]
		\begin{description}
		    \item[Function] Bus stops in Helsinki have a small data sender which is
		usually attached to the stop sign. It is equipped with a small antenna
		that sends its stop code in broadcast in the nearby area of 10mt, so that 
		the bus can listen to the nearby signal and detect when it has reached
		a certain stop. 
			\item[The power source] of this device requires to be always on, so it 
			takes the power from a battery and from solar power when possible.
			The device requires little power, so the battery has to be changed in
			a yearly basis. When the battery is out of energy, a signal is
			sent to the bus and then forwarded to the server. The battery
			should operate between -10°c and 40°c, and resist temperatures until
			-30 in exceptional situations.
			\item[The control unit] is a simple micro-controller because it has
				the simple function of sending the same data. It has also the
				be cheap and be contained in a safe cover to preserve itself
				from the external agents (e.g weather, vandalism).
		\end{description}
	\item[The on-board computer] is embedded in the bus, and is managed by the
		driver. It has two important functional aspects:
		\begin{itemize}
			\item Acts as a gateway between the sensors on the bus and the data
				from the sensors, such as the stop antenna and the passengers'
				card reader.
		    \item Keeps a log of the current driver and route. 
		\end{itemize}
		Usually the driver logs-in with her credentials before to start the turn
		and indicates the route number before to start the route. The data is
		constantly interchanged between the computer and the HSL servers.
		\begin{description}
			\item[Connection] Since HSL operates in the Helsinki metropolitan area, hence not only in
		zone A and B where the public WI-FI could be used, but also in zones C
		and D, where the population density is lower, the most reliable
		connection would be mobile internet, which is covered 
		    \item[The control unit] is a fully-functional low-power, low-energy 
				computer. It has to have enough computational power to handle
				the keyboard input and enable the real-time transmission of
				data with the company servers. 
			\item[The power source] is the bus engine, so there are not energy
				constraints.
		\end{description}
\end{description}

% section smart_traffic (end)


\medskip

\printbibliography


\doclicenseThis

\end{document}
