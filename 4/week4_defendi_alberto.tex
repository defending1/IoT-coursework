\documentclass[a4paper, 12pt]{article}
\usepackage[
    type={CC},
    modifier={by},
    version={4.0},
]{doclicense}                
\setlength{\parindent}{6mm}
\usepackage{amsmath,amsthm,amssymb,amsfonts, mathtools}
\usepackage{graphicx}
\usepackage[margin=1.5in]{geometry}    % For margin alignment
\usepackage[english]{babel}
\usepackage[utf8]{inputenc}
\usepackage{enumerate}
\usepackage[shortlabels]{enumitem}
\usepackage{pdfpages}
\usepackage[ruled, lined, linesnumbered, commentsnumbered, longend]{algorithm2e}
\usepackage{arevmath}     % For math symbols
\usepackage{exercises}
\usepackage[ruled, lined, linesnumbered, commentsnumbered, longend]{algorithm2e}
\DeclarePairedDelimiter{\ceil}{\lceil}{\rceil}
\SetKwInput{KwInput}{Input}                % Set the Input
\SetKwInput{KwOutput}{Output}              % set the Output
\usepackage[
backend=biber,
style=alphabetic,
sorting=ynt
]{biblatex}
\addbibresource{source.bib}

\usepackage{fancyhdr}
\pagestyle{fancy}
\fancyhf{}
\fancyhead[LE,RO]{Alberto Defendi}
\fancyhead[RE,LO]{IoT: week 4 solutions}
\fancyfoot[LE,RO]{\thepage}

\title{Problem solutions 4}
\author{Alberto Defendi - alberto.defendi@helsinki.fi}
\date{\today}
\begin{document}
\maketitle

\section{Solutions} % (fold)
\label{sec:solutions}
\begin{exercise}

\begin{algorithm}[h]
\caption{Monitoring}\label{alg:two}
% Set Function Names
\SetKwFunction{FDiscovery}{Discovery}
\SetKwProg{Fn}{Function}{:}{}
\KwInput{knowNodesList: list of known nodes}
\Fn{\FDiscovery{knowNodesList}}{
	$stop = 0$\;
	\While{$stop \neq 1$}{
		connection = broadcastAdvertise("device-name", "message") \;

		\If {$connection.replies() \neq \emptyset$} {
			connectNodes(nearestNode.getKnownNodesList(), connectedNodesList)\;
		}
	}
}
\SetKwFunction{FKeepAlive}{KeepAlive}
\Fn{\FKeepAlive{knownNodesList}}{
	counter = 0\;
	\While{counter is 5}{
		response = HBRQ(knownNodeList)\;
		\If{response = HBRS}{
			counter = 0\;
		} 
		\Else{
			counter = counter + 1\;
		}
	}
}
\end{algorithm}

The network has to be delay tolerant, since the animals tend to move and the
communication with the infrastructure might be postponed, if the infrastructure
is not available.
\end{exercise}


\begin{exercise}
I chose application 3 and 4.\\
3\\
\begin{enumerate}
    \item 
		\begin{itemize}
		    \item Cellular, since the car has an onboard computer that exchanges
				data of at least 20 MB/s for downloading the maps, and
				the advantage of a good battery capacity. LTE, 5G or 6G may be used.
			\item The car is connected to wireless radios and exchanges data
				with
				the cloud.
			\item Wide network range since the LTE radios may be
				distant.
		\end{itemize}
	\item I would use HTTPS since the data is exchanged via a cellular
		connection. This exposes data to other users, and possible attacks such
		as 
		man-in-the middle.  
	\item Data can be uploaded to the server while the car is connected to a
		WiFi station at home. Here the car can download maps of the
		frequently visited areas, update the police database and to push
		sensor's data to the server, if it was not done while the app was
		functioning.
		This makes possible to save the limited connection rate, and save the
		battery of the car. It can be considered to evaluate the car's condition
		offline using the car's computer to reduce the delays with the server,
		since the sensors re-calibration and the issues are operating in real-time.
		This doesn't increase the production costs excessively, since the
		on-board computer has enough computational power, if it's able to show
		maps.
\end{enumerate}
4\\

\begin{enumerate}
	\begin{itemize}
		\item For the end devices we use ZigBee, since it has a wide range, and
			sufficient data-rate. For the gateway and the server, we use a cellular
			connection, since the amount of data can reach a rate of some MB/s.
	    \item Mesh network for end devices and client-server for the gateway and
			cloud.
		\item The range should be between 10-100m, since the sensor can be
			located distantly.
	\end{itemize}
	\item Zigbee for the edge devices and HTTPS for the gateway-server.
    \item The edges must offload data the earliest as possible, since the
		sensor's control units are low-end devices, with power, memory, and
		energy constraints. Consequently, we can't do any data analysis on
		edges. Furthermore, we have the advantage that the IoT can
		be connected to a high-speed connection with the cloud. The gateway could
		reorganize the data, labeling it with each node name before sending it.
\end{enumerate}
\end{exercise}


\begin{exercise}

We chose article \cite{10.1145/3191756}.
\begin{enumerate}[(i)]
\item BLE because the device is closer than 10m to the gateway and has low
	power consumption.
\item For edge-fog is a PAN, and an area of 10m suffices. The
	smartphone-cloud are at distance of several Km.
\item From fig. 6, W-Air consist in a sensor that
	communicates with a smartphone using Bluetooth. The smartphone
	exchanges data with the cloud using wireless or LAN connection. 
	In a simplified way, the sensor collects the data from the sensors, 
	the smartphone performs some calibrations on the data, and the cloud is
	responsible for the data analysis.
	This model 
	could be viewed as a FC architecture, where the sensor is the edge, the
	smartphone as the fog, and the server as the cloud.
\item COAP because the device has low memory, it needs to continuously exchange the
	measurements with a smartphone. It fits here because the sensor should
	only report data to the smartphone in a publish/subscribe way (where the
	smartphone is the central entity), and a connection-less protocol like
	UDP works since loosing some measurements may not be problematic. 
	For the smartphone and the server, HTTPS can be used with
	request/response.
\end{enumerate}
\end{exercise}

% section solutions (end)

\medskip

\printbibliography

\doclicenseThis

\end{document}
