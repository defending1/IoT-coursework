\documentclass[a4paper, 12pt]{article}
\usepackage[
    type={CC},
    modifier={by},
    version={4.0},
]{doclicense}                
\setlength{\parindent}{6mm}
\usepackage{amsmath,amsthm,amssymb,amsfonts, mathtools}
\usepackage{graphicx}
\usepackage[margin=1.5in]{geometry}    % For margin alignment
\usepackage[english]{babel}
\usepackage[utf8]{inputenc}
\usepackage{enumerate}
\usepackage[shortlabels]{enumitem}
\usepackage{pdfpages}
\usepackage{arevmath}     % For math symbols
\usepackage{exercises}
\usepackage[ruled, lined, linesnumbered, commentsnumbered, longend]{algorithm2e}
\DeclarePairedDelimiter{\ceil}{\lceil}{\rceil}
\SetKwInput{KwInput}{Input}                % Set the Input
\SetKwInput{KwOutput}{Output}              % set the Output
\usepackage[
backend=biber,
style=alphabetic,
sorting=ynt
]{biblatex}
\addbibresource{source.bib}

\usepackage{fancyhdr}
\pagestyle{fancy}
\fancyhf{}
\fancyhead[LE,RO]{Alberto Defendi}
\fancyhead[RE,LO]{IoT: week 3 solutions}
\fancyfoot[LE,RO]{\thepage}

\title{Problem solutions 3}
\author{Alberto Defendi - alberto.defendi@helsinki.fi}
\date{\today}
\begin{document}
\maketitle

\section{Solutions} % (fold)
\label{sec:solutions}

\begin{exercise}
The duty cycle would be:
\begin{table}[htpb]
    \centering
    \caption{Duty cycle}
    \label{tab:label}
    \begin{tabular}{c | c}
		Second (s) & Action\\
	\hline
	0 & Measure temperature and humidity, suddently start and wake up MOX.\\
	4 & Operate LSP.\\
	5 & MOX and LSP are now heated. Calibrate sensors. \\
	6 & Measure LSP and MOX, then start MOX sleeping cycle.\\
	16 & MOX in sleep mode, end of the cycle.\\
    \end{tabular}
\end{table}\\
It is important to measure temperature and humidity before MOX starts, to avoid
the surrounding air to be heated up by that sensor, which has to reach high
temperatures. 

\end{exercise}

\begin{exercise}

		\begin{table*}[htpb]
    \centering
	\caption{Sensor classification (1)}
    \label{tab:label}
    \begin{tabular}{|c | c|}
		\hline
		Name & Category\\
		\hline
		Temperature (thermometer) & Passive, contact (air), non-visual,
		exteroceptive.\\
		\hline
		Temperature (camera) & non-contact, proprioception, visual, active\\
		\hline
		Relative humidity & Passive, contact (soil), non-visual, exteroceptive.\\
		\hline
		Air pressure & Passive, contact (air), non-visual, exteroceptive.\\
		\hline
		Gas and particulate matter (IR) & non-contact, proprioception, visual, active\\
		\hline
    \end{tabular}
\end{table*}
\begin{table*}[htpb]
    \centering
	\caption{Sensor classification (2)}
    \label{tab:label}
    \begin{tabular}{|c | c|}
		\hline
		Name & Category\\
		\hline
		Infrared & Non-contact, proprioception, visual, active.\\
		\hline
		Sound & Exteroceptive, passive, contact (sound waves), non-visual.\\
		\hline
    \end{tabular}
\end{table*}

\textbf{Duty cycle}
\begin{enumerate}
	\item We use a photoacustic $CO$ sensor (does not require calibration),
		a MOX sensor for  $NO_2$ and a LSP
		sensor for the particulate matter. We have a similar duty cycle as that
		of task 1.\\
	\begin{table}[htpb]
		\centering
		\caption{Duty cycle}
		\label{tab:label}
		\begin{tabular}{c | c}
			Time & Action\\
		\hline
		0 & Measure temperature and humidity and air pressure, then wake up MOX.\\
		4 & Operate LSP for 1s, measure temperature, humidity and air pressure.\\
		5 & MOX and LSP are now heated. Calibrate sensors.\\
		5 &  Shut down
		temperature, humidity and air pressure.\\
		6 & Measure LSP, MOX and $CO$. \\
		6 & Start MOX sleeping cycle and shut down LSP, $CO$ sensor.\\
		16 & MOX, LSP in sleep mode. Check values, compute and save next wake up
		time. \\
		\end{tabular}
	\end{table}\\
		In the last step, if the values are under
		the threshold value, the sensors are reactivated the next ten minutes
		in this case, the cycle will last 10 minutes, plus time to warm up MOX.
		If not, the sensors are reactivated in two minutes.  
	\item The criticality of recording cow moves, is that we could have noisy
	recordings due to other agents moving in the environment. For this we
	should use the IR to detect movement, and the noise detector to check, if
	the moved agent is the desired animal. \\
	The passive infrared sensors are always on and located in
	strategic locations, similarly as is done for burglar alarms. Ideally they
	should cover all the area where the animals pass. After one IR sensor is triggered, the nearest
	sound sensor is awaken and starts recording to establish if the desired
	animal passed near the sensor. Between 10PM and 5AM, the device should not
	trigger, if the IR detects another movement and the last registration
	was recorded less than 10 minutes ago. 
\end{enumerate}

\textbf{Pipeline}
\begin{enumerate}[(a)]
    \item The input are the IR sensor trigger, which wakes up the sound sensor
		that records the noise, inferences the sound by classifying it with 
		the know data of possible sounds and
		returns a Boolean value.
	\item Classification, since the given data is trained on a high-end
		computer and then we deploy the knowledge on a low-end system, which
		matches a recording against the model. 
	\item Various sounds of the animals can be recorded in the preceding months
		and those could be edited at computer to introduce common atmospheric
		agents. We can then train a model using an unsupervised network that
		matches our recordings against known sounds of that animal. 
	\item We could use a Bayesian network to establish a probability that the
		sound matches the given data. Implementing a more sophisticated neural
		network or a vector machine would be hard, due to the reduced size of
		the device and of the difficulty to use a internet connection in a
		forest.
\end{enumerate}


\end{exercise}

\begin{exercise}
Analyzing \cite{10.1145/1555816.1555834}.
\begin{description}
    \item[Ground data] pre-collected and classified data is used from third-party
		sources. As reported, "SoundSense leverages the existing body of work on 
		acoustic signal processing and classification (e.g., [24] [34] [21] [14]) 
		and takes a systems design approach to realizing the first such system for 
		a mobile phone".
	\item[Metrics] First, the sound is classified and as one of the three
		category: voice, music and ambient sound. Second, in case of voice and music,
		the system can do a finer discrimination. Third, the agent learns in an
		unsupervised way the ambient sound. 
		SoundSense does not aim to identify all the sound, instead, it aims to
		identify only the most useful sounds.
		When SoundSens detects a new sound, it prompts the user to label this
		sound, asking to provide a textual description.
	\item[Improvements] 
		\begin{description}
			\item[Labeling] The labeling process proposed is not perfect, since 
				the implementation is left to the OS manufacturer.
		    \item[Microphone] as mentioned by the article, the microphone
				recordings become inaccurate when it's obstructed, for instance
				when the phone is in the pockets. One solution can be
				using external microphones, such as the one of headset or of the
				linked smart assistant. The analysis part of the article could
				be extended when using one of these auxiliary devices. Another
				problem is the limited microphone on smartphone, which is
				designed for human voices and not for ambient sounds. Hopefully,
				modern smartphone can integrate ambient microphones. 
			\item[Potential risks] If misused or in case of data breaches from
				the smartphone, this system could leak sensitive information. It
				is important that it is implemented in the OS alongside
				appropriate security features. 
		\end{description}
\end{description}    
\end{exercise}

% section solutions (end)

\medskip

\printbibliography

\doclicenseThis

\end{document}
